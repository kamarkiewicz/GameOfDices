\documentclass[11pt,a4paper]{article}
\usepackage[cm]{fullpage} % wąskie marginesy 
\usepackage{polski}
\usepackage[utf8]{inputenc} 

\title{Game of Dices \\ \small{Specyfikacja funkcjonalna}}
\author{Kamil Markiewicz}

\begin{document}
\maketitle

% \begin{abstract}
% Głównymi założeniami jest . 
% \end{abstract}

\section{Opis ogólny}\label{sec:general}
Przedmiotem specyfikacji jest strategiczna gra turowa. Celem gracza jest kontrolowanie wszystkich elementów mapy, poprzez zdobycie ich w walce. System walki w całości bazuje na rzutach kośćmi. Gra przeznaczona jest na~platformy Windows, i~Linux. Możliwa obsługa systemu Mac~OS~X.

\section{Użyte biblioteki}\label{sec:libs}
Aplikacja korzysta z wieloplatformowych, ogólnodostępnych bibliotek, takich jak CSFML (port C biblioteki SFML służącej do obsługi grafiki, dźwięku, sieci itp.) oraz interpretera języka Lua.

\section{Źródło danych wejściowych}\label{sec:input}
Przewidziana jest obsługa myszki oraz klawiatury. Aplikacja ma również dostęp do plików zawierających ranking graczy i konfigurację.

\section{Dane wyjściowe}\label{sec:output}
Stany gry rysowane są w oknie na ekranie monitora. Dane, które mogą przydać~się przy~kolejnym uruchomieniu aplikacji są~zapisane w~pliku najlepiej w~postaci jawnej (tzn. łatwo dostępnej z poziomu np. notatnika).

\section{Scenariusz działania}\label{sec:script}
Gra przeznaczona jest dla od dwóch do sześciu graczy.
Po uruchomieniu programu, nalezy wcisnąć przycisk "Nowa Gra", aby~zainicjować rozgrywkę. Wygenerowana zostaje plansza, zawierająca pola o~losowej ilości kostek 6-ściennych.
Gra~polega na toczeniu bitew pomiędzy polami graczy, których wynik zależny jest od~ilości oczek wyrzuconych w czasie walki. Jedyną możliwością przejęcia pól przeciwnika jest atak. Przejęcie następuje wtedy i tylko wtedy, gdy uzyskamy w~rzucie więcej oczek niż broniący terenu gracz. W~przeciwnym wypadku na~polu, z~którego atakował agresor pozostaje jedna kostka. Nie~ma ograniczenia ilości walk na~turę. Po~każdej turze do losowych terenów gracza, dodawana jest liczba kostek, równa największemu skupisku jego terenów. Celem gry~jest przejęcie wszystkich terytoriów na~planszy. Gracz, który wygra, na~końcu jest informowany o~tym fakcie odpowiednim powiadomieniem.

\end{document}
